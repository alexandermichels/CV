\documentclass{acmresume}
%-----------------------------------------------------------

\begin{document}
	
	\begin{multicols}{2}
		% \vspace*{.01cm}
		\textbf{\huge Alexander C. Michels}\\ \columnbreak
        %\hfill\href{tel:17167530414}{\faPhone~\textbf{+1 (716) 753 0414}} \\
        \hfill\href{mailto:michels9@illinois.edu}{\textbf{\faEnvelope~michels9@illinois.edu}} \\
        \hfill\href{http://alexandermichels.github.io}{\faGlobeAmericas~\textbf{alexandermichels.github.io}} \\
        % \hfill\href{https://github.com/alexandermichels}{\faGithub~\textbf{github.com/alexandermichels}}
	\end{multicols}
    \vspace*{-.25cm}

\resheading{Summary}

Data scientist with experience in software development, system adminstration, and DevOps. Led software projects with thousands of users, managed software developers, and ran workshops. Experience working in academia, insurance, and media on large, interdisciplinary teams.

\vspace*{.1cm}
	
\resheading{Education}
	
		\Edu{Ph.D. in Informatics}{Jun 2019-May 2025 (\textit{expected})}{University of Illinois Urbana-Champaign}
		
		\Edu{M.S. in Geography}{Aug 2022-May 2024}{University of Illinois Urbana-Champaign}
		
		\Edu{B.S. in Mathematics and Financial Economics}{Aug 2015-May 2019}{Westminster College}
	
% \vspace*{-.15cm}
\resheading{Experience}
	
		\Position{CyberGIS and CyberInfrastructure Researcher}{Champaign, IL}{\href{https://cybergis.illinois.edu/}{CyberGIS Center} \& \href{http://www.cigi.illinois.edu/}{Geospatial Information Laboratory (CIGI)}}{Jun 2019-Present}
        \begin{titemize}
            \titem Lead developer on \href{https://github.com/cybergis/cybergis-compute-python-sdk}{CyberGIS-Compute} (75 users; Typescript, Python) and \href{https://cybergisx.cigi.illinois.edu/}{CyberGIS-Jupyter} (1368 users; Docker, Docker Swarm, Linux, Bash, SLURM, Globus, Ansible, Kubernetes).
            \titem Managed 6 student research programmers; interviewed and hired students and full-time staff.
            \titem Led workshops with 50+ participants and organized conference symposiums/sessions.
            \titem Analyzed spatial Big Data using Bash, HPC, Python, Machine Learning (ML), and SQL.
            \titem Published 16 articles and presented at 20+ conferences, garnering 200+ citations.
        \end{titemize}

		\Position{Systems Administrator and Software Engineer}{New Wilmington, PA}{\href{http://titanradio.net/}{Titan Radio} and \href{https://www.wcn247.com/}{WCN 24/7}}{May 2018-May 2019}
        \begin{titemize}
            \titem{Deployed Linux servers and VirtualBox VMs running software for news and radio staff.}
            \titem{Installed servers, upgraded operating systems, managed software, and automated workflows.}
        \end{titemize}
		
		\Position{Information Technology Intern}{New Castle, PA}{\href{http://treloaronline.com/}{Treloar \& Heisel}}{Sept 2018-Dec 2018}
        \begin{titemize}
            \titem{Cleaned and upgraded database for sales and wrote apps in Java, Python, and Visual Basic.}
        \end{titemize}

        \Position{Data Scientist}{Los Angeles, CA}{\href{https://www.praedicat.com/}{Praedicat, Inc.} \& \href{http://www.ipam.ucla.edu/}{Institute for Pure and Applied Mathematics at UCLA}}{June 2018-Aug 2018}
        \begin{titemize}
            \titem{Built toolkit to crawl and parse millions of webpages for insurtech business insights.}
        \end{titemize}

\vspace*{.15cm}
\resheading{Awards}

        \begin{longtable}{p{0.08\linewidth} p{0.92\linewidth}}
            
            2024 & SDOH \& Place Fellowship, Healthy Regions \& Policies Lab \\
        
            2023 & PI (Co-PI: Dr. Shaowen Wang) (400,000 credits). Advanced Cyberinfrastructure Coordination Ecosystem: Services \& Support (ACCESS) Explore Allocation for “SPACTS: a spatial partitioning algorithm for computing travel-time zones at scale” (CIS230031) \\

            % 2023 & Teacher Ranked as Excellent By Their Students,
            % Center for Innovation in Teaching \& Learning\\
    
            % 2023 & Student of the Year 2022, CyberGIS Center\\
                    
            2022 & \href{https://www.sesync.org/project/graduate-pursuits-request-for-proposals/financial-opacity-and-challenges-to-forest}{Graduate Research Fellow}, National Socio-Environmental Synthesis Center (SESYNC)\\
    
            2020 & Computational Research Techniques Fellow, Texas Advanced Computing Center (TACC)\\
    
            2020 & First Place, Robert Raskin Student Paper Competition, Cyberinfrastructure Group, American Association of Geographers (AAG) \\
    
        \end{longtable}
    
    
\vspace*{-.25cm}
\resheading{Skills}

        \begin{description}[topsep=0pt,itemsep=0pt]
        	\desc[Data Science:] Big Data, G.I.S., Network Science, Git, ML/AI, Parallel Programming
        	
        	\desc[Languages:] Python, Javascript, Typescript, Bash, SQL (Postgres, PostGIS), HTML, C++, Java
        	
        	\desc[Technologies:] Ansible, AWS, Docker, Hadoop, Kubernetes, OpenStack, Terraform
        	
        	\desc[Operating Systems:] Linux (Debian, Ubuntu, CentOS, Scientific), Windows, Apple
        \end{description}

	
\end{document}
