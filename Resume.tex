\documentclass{acmresume}

%-----------------------------------------------------------

\begin{document}
	
	\begin{multicols}{2}
		\vspace*{.1cm}
		\textbf{\Huge Alexander C. Michels}\\ \columnbreak
        \hfill\href{mailto:alexandercm4297@gmail.com}{\textbf{\faEnvelope~michels9@illinois.edu}} \\
        \hfill\href{http://alexandermichels.github.io}{\faGlobeAmericas~\textbf{alexandermichels.github.io}} \\
        \hfill\href{https://github.com/alexandermichels}{\faGithub~\textbf{github.com/alexandermichels}}
	\end{multicols}
	
	\resheading{\faGraduationCap~Education}
	
		\Position{University of Illinois at Urbana-Champaign}{Champaign, IL}{Ph.D. in Informatics}{June 2019-Present}
		\begin{titemize}
			\item{Pursuing the Spatial Informatics concentration under Dr. Shaowen Wang}
		\end{titemize}

		\Position{Westminster College}{New Wilmington, PA}{Bachelor of Science in Mathematics and Financial Economics}{August 2015 - May 2019}
        \begin{titemize}
            \item Honors Thesis: ``Capturing the Predictive Power of Cortical Learning Algorithms''
            \item{Minor in Computer Science | Graduated Cum Laude | Honors in Computer Science and Math | 3.7 GPA}
        \end{titemize}
	
	\resheading{\faBriefcase~Experience}
	
		\Position{CyberGIS and CyberInfrastructure Researcher}{Champaign, IL}{\href{https://cybergis.illinois.edu/}{CyberGIS Center and CyberInfrastructure} \& \href{http://www.cigi.illinois.edu/}{Geospatial Information Laboratory (CIGI)}}{June 2019-Present}
        \begin{titemize}
            \item Building cyberinfrastructure using Docker Swarm, Hadoop and Kubernetes clusters. I manage an undergraduate research assistant and maintain the development branch of \href{https://cybergis.illinois.edu/project/cybergis-jupyter/}{CyberGIS-Jupyter}
            \item Programming spatially-explicit models for disease and land-use change
        \end{titemize}
		
		\Position{Data Scientist and Informatics Reseacher}{Los Angeles, CA}{\href{http://www.ipam.ucla.edu/}{Institute for Pure and Applied Mathematics at UCLA} \& \href{https://www.praedicat.com/}{Praedicat, Inc.}}{June 2018 -August 2018}
        \begin{titemize}
            \item{Worked for Praedicat, Inc. automating information extraction, classification, and aggregation from web data for business profiling of over 52,600 companies and corporate entities.}
            \item{Worked for IPAM to develop a novel algorithm for computational fact-checking on knowledge graphs and a self-supervised machine learning algorithm for sentence importance which outperformed TF-IDF.}
        \end{titemize}

		\Position{Systems Administrator and Software Engineer}{New Wilmington, PA}{\href{http://titanradio.net/}{Titan Radio} and \href{https://www.wcn247.com/}{WCN 24/7}}{May 2018 - May 2019}
        \begin{titemize}
            \item{Responsible for the technology required to keep the radio and television station operating.}
        \end{titemize}
		
		\Position{Information Technology Intern}{New Castle, PA}{\href{http://treloaronline.com/}{Treloar \& Heisel}}{September 2018 - December 2018}
        \begin{titemize}
            \item{Maintained databases and wrote applications for company use in Java, Python, and Visual Basic.}
        \end{titemize}

		\Position{Computational Finance Research Assistant}{New Wilmington, PA}{Dr. Charles Shaffer}{January 2017 - May 2018}
        \begin{titemize}
            \item Integrated cryptocurrency trading into Dr. Shaffer's algorithmic currency trading application.
        \end{titemize}
		
	
		\resheading{\faCodeBranch~Projects}
        
        \Project{CyberGIS-Jupyter}{https://cybergis.illinois.edu/project/cybergis-jupyter/}
        \begin{titemize}
            \item Platform for reproducible spatial data science using JupyterHub on DockerSwarm and Kubernetes
            \item Responsible for development using K8s on Openstack.
        \end{titemize}
		
		\Project{Where COVID-19}{https://wherecovid19.cigi.illinois.edu/}
		\begin{titemize}
			\item Dashboard for displaying up-to-date COVID-19 statistics.
			\item Wrote the pipeline to calculate spatial accessibility to medical resources. DOI: \href{https://doi.org/10.1186/s12942-020-00229-x}{10.1186/s12942-020-00229-x}
		\end{titemize}
	
        \GithubProject{MintSetup}{https://github.com/alexandermichels/MintSetup}
        \begin{titemize}
            \item Python application with PyQt5 GUI that produces Bash scripts
            \item Allows users to produce customized Bash scripts for installing applications and removing pre-installed software
        \end{titemize}
    
    
    		\resheading{\faCogs~Skills}
        \begin{description}[topsep=1pt,itemsep=1pt]
            \desc[\faChartLine \ Data Science:] Big Data, G.I.S., Machine Learning, Parallel Programming, Network Science
            \desc[\faCode \ Languages:] Python (\& Cython), Java, C++, Bash, R, HTML, CSS, XML, Javascript
            \desc[\faServer~Technologies:] Ansible, Docker (Swarm), Hadoop (HDFS/Spark/Yarn), Kubernetes
            \desc[\faDesktop~Operating Systems:] Linux (esp. CentOS, Gentoo, Mint, Raspbian, \& Ubuntu), Windows
        \end{description}

	
\end{document}
