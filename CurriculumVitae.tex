\documentclass{acmcv}
\fancyfoot[L]{Alexander C. Michels} % other info in "inner" position of footer line
\fancyfoot[R]{Updated: \today}
\addbibresource{Bib.bib}
%-----------------------------------------------------------
\renewcommand{\arraystretch}{1.25}

\begin{document}

\begin{center}
    \fontsize{14}{16}\textbf{\uppercase{Alexander C. Michels}}\\
    \vspace*{.5cm}


    \fontsize{12}{14}\textbf{Curriculum Vitae}
    \vspace*{.25cm}
\end{center}

\resheading{Contact Information}
\vspace*{-0.5cm}

\begin{multicols}{2}
    \textbf{Office \& Mailing Address}\\
    The University of Texas at Dallas\\
    800 W. Campbell Road, GR 31\\
    Richardson, TX 75080\\
    
    \textbf{Other}\\
    Telephone: 1 972-883-4712\\
    Email: \href{mailto:Alexander.Michels@utdallas.edu}{Alexander.Michels@utdallas.edu}\\
    Webpage: \href{http://alexandermichels.github.io}{alexandermichels.github.io}
    
\end{multicols}


% \resheading{\faBookOpen~Research Interests}
\resheading{Research \& Teaching Interests}
	\vspace*{-0.5cm}
	\begin{multicols}{2}
		\begin{titemize}
			%\item Agent-Based Models (ABM)
			% \item CyberInfrastructure \& HPC
			\item CyberGIS \& GIScience
			% \item Environmental Data Science \& Justice
			\item Health Geography
			% \item Network Science
			\item Spatial Analysis \& Statistics
            % \item Natural Hazards \& Disasters
			% \item Transportation Geography
            \item Urban Informatics
		\end{titemize}
	\end{multicols}


\resheading{Professional Experience}
	\vspace*{-0.5cm}
    \begin{longtable}{p{0.1\linewidth} p{0.9\linewidth}}
        2025 & \textbf{Assistant Professor} \newline School of Economic, Political and Policy Sciences, The University of Texas at Dallas \\

        2025 & \textbf{Postdoctoral Research Associate} \newline Department of Geography and Geographic Information Science, University of Illinois Urbana-Champaign  \\

        2023-25 & \textbf{Research Assistant}, CyberGIS Center for Advanced Digital and Spatial Studies, University of Illinois Urbana-Champaign  \\

        2023 & \textbf{Teaching Assistant} \newline Department of Geography and Geographic Information Science, University of Illinois Urbana-Champaign  \\

        2019-23 & \textbf{Research Assistant} \newline CyberGIS Center for Advanced Digital and Spatial Studies, University of Illinois Urbana-Champaign  \\
    \end{longtable}

% \resheading{\faGraduationCap~Education}
\resheading{Education}
	\vspace*{-0.5cm}
    \begin{longtable}{p{0.1\linewidth} p{0.9\linewidth}}
        Ph.D. & University of Illinois Urbana-Champaign, Urbana, IL. 2025 \newline Informatics, Advisor: Shaowen Wang \\

        M.S. & University of Illinois Urbana-Champaign, Urbana, IL. 2024 \newline Geography\\

        B.S. & Westminster College, New Wilmington, PA. 2019 \newline Mathematics and Financial Economics\\
    \end{longtable}


% \resheading{\faNewspaper~Publications}
\pubheading{Publications}{https://scholar.google.com/citations?user=EBWPoAwAAAAJ}
		\vspace{-.6cm}
        \nocite{*}
        \newrefcontext[labelprefix=J]
		\printbibliography[title={\normalsize {Journal Articles}},type=article, keyword=journalarticle]
		\vspace{-.8cm}
        \newrefcontext[labelprefix=R]
        \printbibliography[title={\normalsize Journal Articles Under Review / Pre-Prints}, type=article, keyword=inreview]
		\vspace{-.8cm}
        \newrefcontext[labelprefix=C]
        % \nocite{*}
		\printbibliography[title={\normalsize {Peer-Reviewed Conference Papers}},type=inproceedings, keyword=confpaper]
		\vspace{-.8cm}
        \newrefcontext[labelprefix=A]
        % \nocite{*}
        \printbibliography[title={\normalsize {Conference Abstracts}}, keyword=extabs]
		%\vspace{-.75cm}
		%\printbibliography[title={Submitted/Under Review},type=unpublished]
	
% \pagebreak
% \resheading{\faDollarSign~Funding \& Grants}
% \resheading{Funding and Grants}
\resheading{Research Funding}

\begin{longtable}{p{0.1\linewidth} p{0.9\linewidth}}
    2024 & Social Determinants Of Health \& Place Fellowship, Healthy Regions \& Policies Lab \\

    2023 & PI (Co-PI: Shaowen Wang), 400,000 credits. Advanced Cyberinfrastructure Coordination Ecosystem: Services \& Support (ACCESS) \newline ACCESS Explore Allocation for “SPACTS: a spatial partitioning algorithm for computing travel-time zones at scale” (CIS230031) \\


    2022 & \href{https://www.sesync.org/project/graduate-pursuits-request-for-proposals/financial-opacity-and-challenges-to-forest}{SESYNC Graduate Research Fellow}, National Socio-Environmental Synthesis Center (SESYNC)\\

\end{longtable}


% \pagebreak
% \resheading{\faAward~Awards}
\resheading{Awards}

    \begin{longtable}{p{0.1\linewidth} p{0.9\linewidth}}

        2024 & First Place, Data Visualization Competition, Data Science for Everyone Workshop, Practice and Experience in Advanced Research Computing (PEARC) 2024 \\

        2023 & Teacher Ranked as Excellent By Their Students,
        Center for Innovation in Teaching \& Learning, University of Illinois Urbana-Champaign (UIUC)\\

        2023 & Student of the Year 2022, CyberGIS Center for Advanced Digital and Spatial Studies, UIUC\\

        2020 & Third Place, UIUC GIS Day Virtual Student Poster Competition, Department of Geography \& Geographic Information Systems, UIUC \\

        2020 & Computational Research Techniques Fellowship, Texas Advanced Computing Center (TACC)\\

        2020 & First Place, Robert Raskin Student Competition, Cyberinfrastructure Group, American Association of Geographers (AAG) \\

        2019 & UCGIS Prize for Advances in Geospatial Problem Solving, American Association of Geographers (AAG) / University Consortium for Geographic Information Science (UCGIS)  \\

        % 2018 & Best Robot in Division Prize for Senior Unique Division \newline Trinity Fire Fighting Robot Contest \\

        % 2018 & North America Award for Level 2 \newline Trinity Fire Fighting Robot Contest \\

        % 2017 & Paul E. Brown Memorial Scholarship \newline Westminster College Mathematics Department \\

        % 2017 & COMAP International Math Modeling Competition Honorable Mention \newline COMAP International Math Modeling Competition\\

        % 2017 & Mathematics Book Award \newline Westminster College Mathematics Department
    \end{longtable}


% \resheading{\faChalkboardTeacher~Presentations}
\resheading{Selected Presentations}

    \subheading{Invited Talks}

    \begin{longtable}{p{0.1\linewidth} p{0.9\linewidth}}


        2024 & ``CyberGIS for Scalable Spatial Accessibility Analysis''. \textit{The SDOH \& Place End of Year Celebration}. Chicago, IL. Dec 14, 2024. \\
    \end{longtable}

    % \pagebreak
	\subheading{Conference Talks}
    \begin{longtable}{p{0.1\linewidth} p{0.9\linewidth}}
        2025 & ``Expanding Access to CyberGIS-Compute through Support for Heterogeneous Workflows''. \textit{I-GUIDE Forum 2025: Geospatial AI and Innovation for Sustainability Solutions}. Chicago, IL. Jun 19, 2025\\


        2025 & ``Measuring Temporally Dynamic Spatial Accessibility using Machine Learned Driving Times''. \textit{American Association of Geographers (AAG) Annual Meeting}. Detroit, MI. Mar 28, 2025\\

        2024 & ``Data-Intensive Convergence Science for Analyzing Place-Based Spatial Accessibility''. \textit{I-GUIDE Forum 2024: Convergence Science and Geospatial AI for Environmental Sustainability}. Jackson, WY. Oct 15, 2024\\

        2024 & ``Providing Accessible Software Environments Across Science Gateways and HPC''. \textit{Practice and Experience in Advanced Research Computing (PEARC)}. Providence, RI. Jul 24, 2024\\

        2024 & ``Spatial Accessibility with Machine-Learned Driving Times''. \textit{Social Determinants Of Health (SDOH) \& Place Symposium}. Chicago, IL. Jun 15, 2024\\

        2024 & ``Putting the Area in Catchment Areas: An Areal Approach to Spatial Accessibility Analysis''. \textit{American Association of Geographers (AAG) Annual Meeting}. Honolulu, HI. Apr 16, 2024\\

        2023 & ``Streamlined HPC Environments with CVMFS and CyberGIS-Compute''. \textit{I-GUIDE Forum}. New York City, NY. Oct 6, 2023\\

        2023 & ``An Agent-Based Modeling Approach to Spatial Accessibility''. \textit{I-GUIDE Forum}. New York City, NY. Oct 5, 2023\\

        2023 & ``Impacts of Catchments Derived from Fine-Grained Mobility Data on Spatial Accessibility''. \textit{International Conference on Geographic Information Science (GIScience)}. Leeds, UK. Sept 13, 2023\\

        2023 & ``Exploring Road Infrastructure Inequities Across the Conterminous U.S.''. \textit{American Association of Geographers (AAG) Annual Meeting}. Denver, CO. Mar 25, 2023\\

        2022 & ``SCAMEL: Spatial Accessibility Analysis at Scale''. \textit{American Association of Geographers (AAG) Annual Meeting}. Virtual. Feb 28, 2022\\

        2021 & ``Towards Reproducible Research on CyberGISX with Lmod and Easybuild''. \textit{Gateways}. Virtual. Oct 21, 2022\\

        2020 & ``An Exploration of the Effect of Buyer Preference and Market Composition on the Rent Gradient using the ALMA Framework''. \textit{3rd ACM SIGSPATIAL International Workshop on GeoSpatial Simulation}. Virtual. Nov 3, 2020 \\

        2020 & ``Particle Swarm Optimization for Calibration in Spatially Explicit ABMs''. \textit{American Association of Geographers (AAG) Annual Meeting}. Virtual. Apr 10, 2020 \\

        % 2019 & ``Capturing the Predictive Power of Cortical Learning Algorithms''. \textit{National Conference on Undergraduate Research}. Atlanta, GA. Apr 12, 2019 \\

        % 2019 & ``Computational Fact-Checking through Knowledge Graphs''. \textit{AMS Contributed Paper Session at 2019 Joint Mathematics Meeting}. Baltimore, MD. Jan 18, 2019 \\

        % 2017 & ``Repeated Play Games''. \textit{Mathematics Association of America (MAA), Allegheny Mountain Section Meeting}. Pittsburgh, PA. Apr 7, 2017  \\

        % 2017 & ``Optimizing Throughput, Cost, and Safety in Toll Booth Plazas''. \textit{Pi Mu Epsilon Regional Conference}. Youngstown, OH. Apr 2017 \\
    \end{longtable}

    % \pagebreak
    \subheading{Tutorials \& Workshops}
    \begin{longtable}{p{0.1\linewidth} p{0.9\linewidth}}
        2025 & ``Introduction to the I-GUIDE Platform''. \textit{I-GUIDE Virtual Consulting Office (VCO)}. Virtual. Jul 9, 2025.\\

        2025 & ``Introduction to the I-GUIDE Platform''. \textit{I-GUIDE Forum 2025}. Chicago, IL. Jun 17, 2025.\\

        2025 & ``Calculating Spatial Accessibility in Python with the I-GUIDE Platform''. \textit{I-GUIDE Virtual Consulting Office (VCO)}. Virtual. May 28, 2025.\\

        2024 & ``I-GUIDE Platform'' with Anand Padmanabhan. \textit{I-GUIDE Summer School 2024: Leveraging AI for Environmental Sustainability}. Boulder, CO. Aug 6, 2024.\\

        2023 & ``Geospatial Knowledge Discovery Harnessing Pre-trained Language Models on CyberGISX'' with Zhaonan Wang, Wei Hu, and Anand Padmanabhan. \textit{2023 NSF HDR Ecosystem Conference}. Denver, CO. Oct 17, 2023.\\

        2023 & ``CyberGIS-Compute: Geospatial Middleware for High-Performance Computing'' with Anand Padmanabhan and Shaowen Wang. \textit{I-GUIDE Forum 2023}. New York City, NY. Oct 4, 2023.\\

        2023 & ``CyberGIS-Compute: Geospatial Middleware for Simplifying Access to High-Performance Computing'' with Furqan Baig. \textit{Accelerating Computing for Emerging Sciences (ACES) Workshop 2023}. League City, TX. Jul 15, 2023.\\

        2023 & ``CyberGIS-Compute: Geospatial Middleware for High-Performance Computing''. \textit{Annual Meeting of the American Association of Geographers (AAG) 2023}. Denver, CO. Mar 24, 2023.\\

        2022 & ``CyberGIS-Compute: Enabling Simplified Access to High Performance Computing for your Geospatial Computation'' with Anand Padmanabhan. \textit{I-GUIDE Virtual Consulting Office (VCO)}. Virtual. Nov. 2, 2022.\\

        2022 & ``CyberGIS-Compute: Geospatial Middleware for Simplifying Access to High-Performance Computing'' with Anand Padmanabhan. \textit{I-GUIDE Virtual Consulting Office (VCO)}. Virtual. Jul 27, 2022. \\
    \end{longtable}
    % \vspace*{-.5cm}
    \subheading{Poster Presentations}

    \begin{longtable}{p{0.1\linewidth} p{0.9\linewidth}}
        2024 & ``Measuring Road Network Equity and Resilience for Evacuations and Natural Hazards''. \textit{I-GUIDE Forum 2024: Convergence Science and Geospatial AI for Environmental Sustainability}. Jackson, WY. Oct 15, 2024 \\

        2024 & ``Measuring Road Network Equity and Resilience for Evacuations and Natural Hazards''. \textit{Geo-Resolution 2024}. St. Louis, MO. Sept 12, 2024 \\

        2023 & ``CyberGIS-Compute: Middleware for Democratizing Scalable Geocomputation''. \textit{2023 NSF HDR Ecosystem Conference}. Denver, CO. Oct 16, 2023 \\

        2021 & ``ScalableAccess: Travel-Time Polygons for Accessibility at Scale''. \textit{UIUC GIS Day}. Champaign, IL. Nov 17, 2021 \\

        2021 & ``Rapidly Measuring Spatial Accessibility of COVID-19 Healthcare Resources: A Case Study of Illinois, USA''. \textit{UIUC School of Earth Society \& Environment (SESE) Research Review}. Champaign, IL. Apr 23, 2021 \\ 

        2020 & ``Effect of Buyer Preference and Market Composition on the Rent Gradient''. \textit{UIUC GIS Day}. Champaign, IL. Nov 18, 2020 \\

        2020 & ``Particle Swarm Optimization for Calibration in Spatially Explicit ABMs''. \textit{UIUC SESE Research Review}. Champaign, IL. Feb 14, 2020 \\ 

        2019 & ``CyberGIS-Jupyter for Spatially Explicit Agent-based Modeling''. \textit{UIUC GIS Day}. Champaign, IL. Nov 13, 2020 \\

        % 2018 & ``Computational Fact-Checking through Knowledge Graphs''. \textit{Undergraduate Research Poster Session at 2019 Joint Mathematics Meeting}. Baltimore, MD. Jan 19, 2019 \\
    \end{longtable}



\resheading{Teaching and Mentoring}

    \subheading{Courses Taught as Instructor of Record}


    \begin{longtable}{p{0.16\linewidth} p{0.84\linewidth}}
        Spring 2023 & \textbf{Business Location Decisions (GGIS/BADM 205)} \newline Department of Geography and Geographic Information Science, UIUC \\

    \end{longtable}
    % \noindent\textcolor{color1}{\rule{\textwidth}{.01cm}}
    % \pagebreak
    \subheading{Undergraduate Student Mentees}

    \begin{longtable}{p{0.16\linewidth} p{0.84\linewidth}}
        2025 & Vaibhavi Srivastava, B.S. in CS+GGIS, UIUC \\
        2025 & Akriti Arora, B.S. in CS, UIUC \\
        2023-2025 & Ian Zhang, B.S. in CS+Math, UIUC \\
        2023-24 & John Speaks, B.S. in CS+Linguistics, UIUC\\
        2023-24 & Jeffrey Huang, B.S. in CS+Statistics, UIUC \\
        2022-23 & Mit Kotak, B.S. in Physics, UIUC\\
        2022-23 & Taylor Ziegler, B.S. in CS, UIUC\\
        2019-22 & Zimo Xiao, B.S. in CS+GGIS, UIUC\\
    \end{longtable}


    % \pagebreak

    \subheading{Certificates and Workshops}

    \begin{longtable}{p{0.16\linewidth} p{0.84\linewidth}}
        May 2023 & Certificate in Foundations of Teaching, The Center for Innovation in Teaching \& Learning, UIUC \\

        Jan 2023 & Graduate Academy for College Teaching, The Center for Innovation in Teaching \& Learning, UIUC \\

    \end{longtable}

    


% \resheading{\faHandsHelping~Service}
\resheading{Service}

    \subheading{Professional Organizations}
    \begin{longtable}{p{0.1\linewidth} p{0.9\linewidth}}

        2022-26 & \textbf{Director}, AAG CyberInfrastructure Specialty Group (CISG) \\

        2021-22 & \textbf{Student Director}, AAG CyberInfrastructure Specialty Group
    \end{longtable}

    \subheading{Conferences and Workshops}
    \begin{longtable}{p{0.1\linewidth} p{0.9\linewidth}}
        2025 & \textbf{Reviewer}, \href{https://i-guide.io/forum/forum-2025/}{Institute for Geospatial Understanding through an Integrative Discovery Environment (I-GUIDE) Forum 2025}\\

        2025 & \textbf{Symposium Co-Organizer}, \href{https://i-guide.io/aag-2025-symposium-on-spatial-ai-data-science-for-sustainability/}{AAG 2025 Symposium on Spatial AI \& Data Science for Sustainability}\\

        2025 & \textbf{Session Chair}, ``Challenges and Opportunities of Spatial Accessibility'' \href{https://iguide.illinois.edu/aag-2024-symposium-on-geospatial-data-science-for-sustainability/}{AAG 2025 Symposium on Spatial AI \& Data Science for Sustainability}\\

        2024 & \textbf{Symposium Co-Organizer}, \href{https://iguide.illinois.edu/aag-2024-symposium-on-geospatial-data-science-for-sustainability/}{AAG 2024 Symposium on Geospatial Data Science for Sustainability}\\

        2024 & \textbf{Session Organizer}, ``Challenges and Opportunities of Spatial Accessibility'' \href{https://iguide.illinois.edu/aag-2024-symposium-on-geospatial-data-science-for-sustainability/}{AAG 2024 Symposium on Geospatial Data Science for Sustainability}\\
        
        2023 & \textbf{Reviewer}, \href{https://iguide.illinois.edu/forum-2023/}{Institute for Geospatial Understanding through an Integrative Discovery Environment (I-GUIDE) Forum 2023}\\

        2023 & \textbf{Symposium Program Co-Chair}, \href{https://iguide.illinois.edu/aag-2023-symposium-on-harnessing-the-geospatial-data-revolution-for-sustainability-solutions/}{AAG 2023 Symposium on Harnessing the Geospatial Data Revolution for Sustainability Solutions}\\

        2023 & \textbf{Session Chair}, ``Data-intensive and Computational Geography,'' \href{https://iguide.illinois.edu/aag-2023-symposium-on-harnessing-the-geospatial-data-revolution-for-sustainability-solutions/}{AAG 2023 Symposium on Harnessing the Geospatial Data Revolution for Sustainability Solutions}\\

        2022 & \textbf{Session Organizer}, ``Computation and Uncertainty of Spatial Accessibility,'' \href{https://cybergis.illinois.edu/aag-symposium-2022/}{AAG 2022 Symposium on Data-Intensive Geospatial Understanding in the Era of CyberGIS}\\
    \end{longtable}

    \subheading{Journal Reviewer}
    \begin{titemize}
        \item Geocarto International, Taylor \& Francis
        \item International Journal of Geographical Information Science (IJGIS), Taylor \& Francis
        \item ISPRS International Journal of Geo-Information, MDPI
        \item Quality \& Quantity, Springer
    \end{titemize}


    % \textbf{\uppercase{Departmental Service}}
    % \begin{longtable}{p{0.16\linewidth} p{0.84\linewidth}}
    %     2022 & \textbf{Program Ambassador}, UIUC Informatics Program \newline Hosted Q\&A sessions for prospective and incoming Informatics students \\
    % \end{longtable}

    % \resheading{\faUsers~Professional Associations}
\resheading{Professional Associations}

\begin{titemize}
    \item American Association of Geographers (AAG)
    \item Association for Computing Machinery (ACM)
    \item Campus Research Computing Consortium (CaRCC)
    \item United States Research Software Engineer Association (US-RSE)
\end{titemize}




\end{document}
