\documentclass{acmcv}
\fancyfoot[L]{Alexander Michels} % other info in "inner" position of footer line
\fancyfoot[R]{\textit{Curriculum Vitae}}
\addbibresource{Bib.bib}
%-----------------------------------------------------------
\begin{document}

	\begin{multicols}{2}
		\vspace*{.15cm}
		\textbf{\LARGE Alexander C. Michels} \\
		\columnbreak
        %\hfill\href{tel:17167530414}{\faPhone~\textbf{+1 (716) 753 0414}} \\
		\hfill\href{mailto:michels9@illinois.edu}{\textbf{\faEnvelope~michels9@illinois.edu}} \\
		\hfill\href{http://alexandermichels.github.io}{\faGlobeAmericas~\textbf{alexandermichels.github.io}} \\
        % \hfill Last Updated: \today \\
		%\hfill\href{https://github.com/alexandermichels}{\faGithub~\textbf{github.com/alexandermichels}}
	\end{multicols}

% \resheading{\faBookOpen~Research Interests}
\resheading{Research Interests}
	
	\begin{multicols}{2}
		\begin{titemize}
			%\item Agent-Based Models (ABM)
			\item CyberInfrastructure \& HPC
			\item CyberGIS \& GIScience
			\item Health Geography
			\item Network Science
			\item Spatial Analysis \& Statistics
			\item Transportation Geography
		\end{titemize}
	\end{multicols}

% \resheading{\faGraduationCap~Education}
\resheading{Education}

    \begin{longtable}{p{0.1\linewidth} p{0.9\linewidth}}
        Ph.D. & University of Illinois Urbana-Champaign, Urbana, IL. 2025 (expected) \newline Informatics, Spatial concentration, \textbf{Advisor:} Dr. Shaowen Wang \\

        M.S. & University of Illinois Urbana-Champaign, Urbana, IL. 2024 (expected) \newline Geography and Geographic Information Science\\

        B.S. & Westminster College, New Wilmington, PA. 2019 \newline Mathematics and Financial Economics\\
    \end{longtable}
    \vspace*{-0.5cm}

% \resheading{\faNewspaper~Publications}
\pubheading{Publications}{https://scholar.google.com/citations?user=EbmZrwYAAAAJ}
		\vspace{-.5cm}
		\nocite{*}
		\printbibliography[title={\normalsize \uppercase{Journal Articles}},type=article]
		\vspace{-.8cm}
		\printbibliography[title={\normalsize \uppercase{Peer-Reviewed Conference Papers}},type=inproceedings, keyword=confpaper]
		\vspace{-.8cm}
        \printbibliography[title={\normalsize \uppercase{Peer-Reviewed Conference Extended Abstracts}}, keyword=extabs]
		%\vspace{-.75cm}
		%\printbibliography[title={Submitted/Under Review},type=unpublished]
	
% \pagebreak
% \resheading{\faDollarSign~Funding \& Grants}
% \resheading{Funding and Grants}
\resheading{Research Grants}

\begin{longtable}{p{0.1\linewidth} p{0.9\linewidth}}
    2023 & PI (Co-PI: Dr. Shaowen Wang), 400,000 credits \newline Advanced Cyberinfrastructure Coordination Ecosystem: Services \& Support (ACCESS) \newline ACCESS Explore Allocation for “SPACTS: a spatial partitioning algorithm for computing travel-time zones at scale” (CIS230031) \\

\end{longtable}


% \pagebreak
% \resheading{\faAward~Awards}
\resheading{Awards}

    \begin{longtable}{p{0.1\linewidth} p{0.9\linewidth}}
        2024 & SDOH \& Place Fellowship, Healthy Regions \& Policies Lab \\

        2023 & Teacher Ranked as Excellent By Their Students,
        Center for Innovation in Teaching \& Learning\\

        2023 & Student of the Year 2022, CyberGIS Center\\

        2022 & \href{https://www.sesync.org/project/graduate-pursuits-request-for-proposals/financial-opacity-and-challenges-to-forest}{SESYNC Graduate Research Fellow}, National Socio-Environmental Synthesis Center (SESYNC)\\

        2020 & Third Place, UIUC GIS Day Virtual Student Poster Competition, UIUC Department of Geography \& Geographic Information Systems \\

        2020 & Computational Research Techniques Fellowship, Texas Advanced Computing Center (TACC)\\

        2020 & First Place, Robert Raskin Student Competition, Cyberinfrastructure Group, American Association of Geographers (AAG) \\

        2019 & UCGIS Prize for Advances in Geospatial Problem Solving, American Association of Geographers (AAG) / University Consortium for Geographic Information Science (UCGIS)  \\

        % 2018 & Best Robot in Division Prize for Senior Unique Division \newline Trinity Fire Fighting Robot Contest \\

        % 2018 & North America Award for Level 2 \newline Trinity Fire Fighting Robot Contest \\

        % 2017 & Paul E. Brown Memorial Scholarship \newline Westminster College Mathematics Department \\

        % 2017 & COMAP International Math Modeling Competition Honorable Mention \newline COMAP International Math Modeling Competition\\

        % 2017 & Mathematics Book Award \newline Westminster College Mathematics Department
    \end{longtable}


% \resheading{\faChalkboardTeacher~Presentations}
\resheading{Selected Presentations}
\vspace*{0.25cm}

    \textbf{\uppercase{Invited Talks}}
    \vspace*{-0.35cm}

    \begin{longtable}{p{0.1\linewidth} p{0.9\linewidth}}
        % 2022 & ``CyberGIS-Compute: Geospatial Middleware for Simplifying Access to High-Performance Computing'' with Dr. Furqan Baig. \textit{Accelerating Computing for Emerging Sciences (ACES) Workshop 2023}. League City, TX. July 15, 2023.\\

        2022 & ``CyberGIS-Compute: Enabling Simplified Access to High Performance Computing for your Geospatial Computation'' with Dr. Anand Padmanabhan. \textit{NSF Institute for Geospatial Understanding through an Integrative Discovery Environment (I-GUIDE) - Virtual Consulting Office}. Virtual. Nov. 2, 2022.\\

        2022 & ``CyberGIS-Compute: Geospatial Middleware for Simplifying Access to High-Performance Computing'' with Dr. Anand Padmanabhan. \textit{NSF Institute for Geospatial Understanding through an Integrative Discovery Environment (I-GUIDE) - Virtual Consulting Office}. Virtual. July 27, 2022. \\

        % 2018 & ``Information Extraction and Aggregation for Business Profiling''. \textit{Institute for Pure and Applied Mathematics}, Los Angeles, CA. August 24, 2018 \\

        % 2018 & ``Information Extraction and Aggregation for Business Profiling''. \textit{Praedicat, Inc.}, Los Angeles, CA. July 31, 2018 \\
    \end{longtable}
    \vspace*{-0.1cm}

    % \pagebreak
	\textbf{\uppercase{Conference Talks}}
    \vspace*{-0.35cm}

    \begin{longtable}{p{0.1\linewidth} p{0.9\linewidth}}
        2023 & ``Impacts of Catchments Derived from Fine-Grained Mobility Data on Spatial Accessibility''. \textit{International Conference on Geographic Information Science (GIScience)}, Leeds, UK, Sept 13, 2023\\

        2023 & ``Exploring Road Infrastructure Inequities Across the Conterminous U.S.''. \textit{American Association of Geographers (AAG) Annual Meeting}, Denver, CO, March 25, 2023\\

        2022 & ``SCAMEL: Spatial Accessibility Analysis at Scale''. \textit{American Association of Geographers (AAG) Annual Meeting}, Virtual, February 28, 2022\\

        2021 & ``Towards Reproducible Research on CyberGISX with Lmod and Easybuild''. \textit{Gateways}, Virtual. Oct 21, 2022\\

        2020 & ``An Exploration of the Effect of Buyer Preference and Market Composition on the Rent Gradient using the ALMA Framework''. \textit{3rd ACM SIGSPATIAL International Workshop on GeoSpatial Simulation}, Virtual. November 3, 2020 \\

        2020 & ``Particle Swarm Optimization for Calibration in Spatially Explicit ABMs''. \textit{American Association of Geographers (AAG) Annual Meeting}, Virtual. April 10, 2020 \\

        % 2019 & ``Capturing the Predictive Power of Cortical Learning Algorithms''. \textit{National Conference on Undergraduate Research}, Atlanta, GA. April 12, 2019 \\

        % 2019 & ``Computational Fact-Checking through Knowledge Graphs''. \textit{AMS Contributed Paper Session at 2019 Joint Mathematics Meeting}, Baltimore, MD. January 18, 2019 \\

        % 2017 & ``Repeated Play Games''. \textit{Mathematics Association of America (MAA), Allegheny Mountain Section Meeting}, Pittsburgh, PA. April 7, 2017  \\

        % 2017 & ``Optimizing Throughput, Cost, and Safety in Toll Booth Plazas''. \textit{Pi Mu Epsilon Regional Conference}, Youngstown, OH. April 2017 \\
    \end{longtable}
    \vspace*{-0.2cm}
    
    \pagebreak
    \textbf{\uppercase{Tutorials \& Workshops}}
    \vspace*{-0.35cm}
    \begin{longtable}{p{0.1\linewidth} p{0.9\linewidth}}
        2023 & ``Geospatial Knowledge Discovery Harnessing Pre-trained Language Models on CyberGISX'' with Zhaonan Wang, Wei Hu, and Anand Padmanabhan. \textit{2023 NSF HDR Ecosystem Conference}. Denver, CO. Oct 17, 2023.\\

        2023 & ``CyberGIS-Compute: Geospatial Middleware for High-Performance Computing'' with Anand Padmanabhan and Shaowen Wang. \textit{I-GUIDE Forum 2023}. New York City, NY. Oct 4, 2023.\\

        2023 & ``CyberGIS-Compute: Geospatial Middleware for Simplifying Access to High-Performance Computing'' with Furqan Baig. \textit{Accelerating Computing for Emerging Sciences (ACES) Workshop 2023}. League City, TX. July 15, 2023.\\

        2023 & ``CyberGIS-Compute: Geospatial Middleware for High-Performance Computing''. \textit{Annual Meeting of the American Association of Geographers (AAG) 2023}. Denver, CO. March 24, 2023.\\

    \end{longtable}
    \vspace*{-0.2cm}

    \textbf{\uppercase{Poster Presentations}}
    \vspace*{-0.3cm}

    \begin{longtable}{p{0.1\linewidth} p{0.9\linewidth}}
        2023 & ``CyberGIS-Compute: Middleware for Democratizing Scalable Geocomputation''. \textit{2023 NSF HDR Ecosystem Conference}, Denver, CO. October 16, 2023 \\

        2021 & ``ScalableAccess: Travel-Time Polygons for Accessibility at Scale''. \textit{UIUC GIS Day}, Champaign, IL. November 17, 2021 \\

        2021 & ``Rapidly Measuring Spatial Accessibility of COVID-19 Healthcare Resources: A Case Study of Illinois, USA''. \textit{UIUC SESE Research Review}, Champaign, IL. April 23, 2021 \\ 

        2020 & ``Effect of Buyer Preference and Market Composition on the Rent Gradient''. \textit{UIUC GIS Day}, Champaign, IL. November 18, 2020 \\

        2020 & ``Particle Swarm Optimization for Calibration in Spatially Explicit ABMs''. \textit{UIUC SESE Research Review}, Champaign, IL. February 14, 2020 \\ 

        2019 & ``CyberGIS-Jupyter for Spatially Explicit Agent-based Modeling''. \textit{UIUC GIS Day}, Champaign, IL. November 13, 2020 \\

        2018 & ``Computational Fact-Checking through Knowledge Graphs''. \textit{Undergraduate Research Poster Session at 2019 Joint Mathematics Meeting}, Baltimore, MD. January 19, 2019 \\
    \end{longtable}

% \pagebreak
% \resheading{\faLaptopCode~Research Experience}

% \vspace*{-0.5cm}
% \resheading{Research Experience}

%     \begin{longtable}{p{0.16\linewidth} p{0.84\linewidth}}
%         2019-Present & \textbf{Research Assistant} \newline \href{https://cybergis.illinois.edu/}{CyberGIS Center} \& \href{http://www.cigi.illinois.edu/}{Geospatial Information Laboratory (CIGI)}\\

%         2020 - 2022 & \textbf{SESYNC Graduate Research Fellow} \newline National Socio-Environmental Synthesis Center (SESYNC)\\

%         2018 & \textbf{Informatics Researcher} \newline Institute for Pure and Applied Mathematics at UCLA / Praedicat, Inc.\\
%     \end{longtable}


% \vspace*{-0.5cm}
% \resheading{\faLaptopCode~Industry Experience}
% \resheading{Industry Experience}

% \begin{longtable}{p{0.16\linewidth} p{0.84\linewidth}}
%     2018 - 2019 & \textbf{Systems Administrator and Software Engineer} \newline \href{http://titanradio.net/}{Titan Radio} and \href{https://www.wcn247.com/}{WCN 24/7}, New Wilmington, PA\\

%     2018 & \textbf{Data Scientist} \newline \href{http://treloaronline.com/}{Treloar \& Heisel}, New Castle, PA\\

%     2018 & \textbf{Data Scientist} \newline Institute for Pure and Applied Mathematics at UCLA / Praedicat, Inc., Los Angeles, CA\\
% \end{longtable}
    
% \pagebreak
% \resheading{\faUniversity~Teaching/Mentoring Experience}
\resheading{Teaching and Mentoring}
\vspace*{0.25cm}

    \textbf{Courses Taught as Instructor of Record}
    \vspace*{-0.3cm}

    % \noindent\textcolor{color1}{\rule{\textwidth}{.01cm}}\vspace*{-0.25cm}
    \begin{longtable}{p{0.16\linewidth} p{0.84\linewidth}}
        Spring 2023 & \textbf{Business Location Decisions (GGIS/BADM 205)} \newline Department of Geography and Geographic Information Science, UIUC \newline
        List of Teachers Ranked as Excellent By Their Students \\

    \end{longtable}
    % \noindent\textcolor{color1}{\rule{\textwidth}{.01cm}}
    % \pagebreak
    \textbf{Undergraduate Student Mentees}
    \vspace*{-0.3cm}
    \begin{longtable}{p{0.16\linewidth} p{0.84\linewidth}}
        2023 - & Ian Zhang\\
    
        2023 - & Jeffrey Huang\\
    
        2023 - & John Speaks\\
    
        2022 - 2023 & Taylor Ziegler, software engineering intern\\
    
        2022 - 2023 & Mit Kotak, pursuing PhD at MIT\\

        2019 - 2022 & Zimo Xiao, pursuing MS at CMU\\
    \end{longtable}

    % \Position{Teaching Assistant and Tutor}{New Wilmington, PA}{Westminster College}{Aug 2015 - Dec 2018}
    % \vspace*{-0.25cm}
    % \begin{titemize}
    % 	\titem Assisted professors in grading and working with students individually for classes covering coursework in Math, Computer Science, and Operations Research.
    % \end{titemize}\vspace*{0.25cm}

    % \Position{Math Tutor}{New Wilmington, PA}{Independent Contractor}{Aug 2016 - May 2018}
    % \vspace*{-0.25cm}
    % \begin{titemize}
    % 	\titem Worked with middle and high-school aged students to expand their Math skills.
    % \end{titemize}

    % \pagebreak
% \resheading{\faUsers~Professional Associations}
\resheading{Professional Associations}
\vspace*{.2cm}

    \textbf{American Association of Geographers (AAG)} \\
    Specialty Groups: \\ \vspace{-.3cm}
    \begin{multicols}{2}
    \begin{titemize}
        \item Applied Geography
        \item Cyberinfrastructure
        \item GIScience \& GIS
        \item Health and Medical Geography
        \item Spatial Analysis and Modeling
        \item Transportation Geography
    \end{titemize}
    \end{multicols}
    \vspace*{-0.2cm}

    \textbf{Association for Computing Machinery (ACM)} \\
    Special Interest Group:  SIGSPATIAL (Spatial Information)


% \resheading{\faHandsHelping~Service}
\vspace*{0.5cm}
\resheading{Service}
\vspace*{0.25cm}

    \textbf{\uppercase{Conferences and Workshops}}
    \begin{longtable}{p{0.16\linewidth} p{0.84\linewidth}}
        2023-2024 & \textbf{Symposium Program Co-Chair}, \href{https://iguide.illinois.edu/aag-2024-symposium-on-geospatial-data-science-for-sustainability/}{AAG 2024 Symposium on Geospatial Data Science for Sustainability}\\
        
        2023 & \textbf{Reviewer}, \href{https://iguide.illinois.edu/forum-2023/}{Institute for Geospatial Understanding through an Integrative Discovery Environment (I-GUIDE) Forum}\\

        2022-2023 & \textbf{Symposium Program Co-Chair}, \href{https://iguide.illinois.edu/aag-2023-symposium-on-harnessing-the-geospatial-data-revolution-for-sustainability-solutions/}{AAG 2023 Symposium on Harnessing the Geospatial Data Revolution for Sustainability Solutions}\\

        2023 & \textbf{Session Chair}, ``Data-intensive and Computational Geography,'' \href{https://iguide.illinois.edu/aag-2023-symposium-on-harnessing-the-geospatial-data-revolution-for-sustainability-solutions/}{AAG 2023 Symposium on Harnessing the Geospatial Data Revolution for Sustainability Solutions}\\

        2022 & \textbf{Session Organizer}, ``Computation and Uncertainty of Spatial Accessibility,'' AAG 2022 Symposium on Data-Intensive Geospatial Understanding in the Era of CyberGIS\\
    \end{longtable}

    \textbf{\uppercase{Journal Reviewer}}
    \begin{titemize}
        \item Geocarto International, Taylor \& Francis
    \end{titemize}
    \vspace*{.25cm}
% \pagebreak
    \textbf{\uppercase{Professional Organizations}}
    \begin{longtable}{p{0.16\linewidth} p{0.84\linewidth}}

        2022-Present & \textbf{Director, AAG CyberInfrastructure Specialty Group (CISG)} \\

        2021-2022 & \textbf{Student Director, AAG CyberInfrastructure Specialty Group}
    \end{longtable}

    \textbf{\uppercase{Departmental Service}}
    \begin{longtable}{p{0.16\linewidth} p{0.84\linewidth}}
        2022 & \textbf{Program Ambassador}, UIUC Informatics Program \newline Hosted Q\&A sessions for prospective and incoming Informatics students \\
    \end{longtable}

% \resheading{\faCogs~Technical Skills}
	
% 	\begin{description}[topsep=0pt,itemsep=1pt]
% 		\desc[\faChartLine~Data Science:] G.I.S., Git, Machine Learning, Parallel Programming, Network Science
		
% 		\desc[\faCode~Languages:] Python, Bash, Javascript, C++, R, SQL
		
% 		\desc[\faServer~Technologies:] Cloud Computing, Docker, Easybuild, Hadoop (HDFS/Spark/Yarn), Kubernetes, OpenStack, Terraform
		
% 		\desc[\faDesktop~Operating Systems:] Linux, Windows
% 	\end{description}


\end{document}
